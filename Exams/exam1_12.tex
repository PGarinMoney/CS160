\documentclass[11pt]{exam}
\usepackage{listings}
\usepackage{pdfsync}

%
%  Created by Brad Miller on 2006-03-07.
%  Copyright (c) 2006 Luther College. All rights reserved.
%
%

\newif\ifpdf
\ifx\pdfoutput\undefined
\pdffalse % we are not running PDFLaTeX
\else
\pdfoutput=1 % we are running PDFLaTeX
\pdftrue
\fi

\ifpdf
\usepackage{subfigure}
\usepackage[pdftex]{graphicx}
\else
\usepackage{graphicx}
\fi

%
%  Update these values for running headers
%
\firstpageheader{\bf\Large CS-151}{\bf\Large Stacks, Queues, and Analysis}{\bf\Large
  2012-03-14 }
\runningheader{CS 151}{}{Exam-1}
\addpoints

\begin{document}

\begin{center} 
  \fbox{\fbox{\parbox{5.5in}{\centering This Exam is being given under
        the guidelines of the \textbf{Honor Code}. You are expected to
        respect those guidelines and to report those who do not.
        Answer the questions in the spaces provided. If you run out of
        room for an answer, continue on the back of the page.  There are
      \numquestions\  questions for a total of  \numpoints\ points.}}}
\end{center} 

% setup standard options for the including code fragments
\lstset{language=Python,numbers=left}

\vspace{0.1in} 
\hbox to \textwidth{Name:\enspace\hrulefill} 

% Questions start here:
\begin{questions}

% \question Suppose you are writing an application to keep the payroll for some company.  Your application consists of \textit{at least} three classes:  \texttt{Person}, \texttt{Manager}, and \texttt{Worker}.  A \texttt{Person}  has a name and year of birth.  A \texttt{Worker} has a department and a monthly salary. A \texttt{Manager} has a department and a salary and a bonus equal to 10\% of their salary.  Hint:  Sometimes it can make your life easier if you introduce an ``extra'' class into your hierarchy to accomodate two classes that are similar.

% \begin{parts}
% 	\part[10] Implement the classes of your inheritance hierarchy.  In particular you should write an efficient constructor for each class, and an \verb~__str__~ method for each class, and a \texttt{compensation} method that returns the total yearly compensation.
% 	\vspace{5in}
% \end{parts}

% \newpage


\question[5] Convert the following expression to its Postfix form.  You do not need to demonstrate the algorithm, just write the final form.
$$A~/~B~*~C~+~D$$
\vspace{1.5in}

\question[5] Show how the stack is used in evaluating the following postfix expression.  Make sure you show the stack at enough stages to convince me that you understand the algorithm.
$$7~13~*~12~8~-~4~+~10~*~+$$
\vspace{1.5in}

\newpage
\question[5] Rank the following functions in order from slowest
growing to fastest.
\begin{itemize}
\item $N$
\item $N log(N)$
\item $1$
\item $log(N)$
\item $N^2$
\end{itemize}

\question[5] Give the Big-O limits for the following Python operations.
\begin{itemize}
\item \texttt{list.pop(0)}
\item \texttt{list.pop()}
\item \texttt{dict[x]}
\item \texttt{list.append()}
\item \texttt{dict[key] = value}
\end{itemize}

\newpage

\question[10] For this question you must implement a \texttt{Stack} class.  However you may \textbf{not} use a Python list in your implementation.  You \textbf{must} use The \textit{Node}  class shown below.   Implement the methods \verb!__init__!, \texttt{push}, \texttt{pop}, and \texttt{isempty}.
\begin{verbatim}
	class Node:
    def __init__(self,initdata):
        self.data = initdata
        self.next = None

    def getData(self):
        return self.data

    def getNext(self):
        return self.next

    def setData(self,newdata):
        self.data = newdata

    def setNext(self,newnext):
        self.next = newnext
\end{verbatim}

\newpage


%Given a singly linked list, how could the list be printed in reverse order in O(n)?


\question Using the following code fragment:
\begin{lstlisting}
for i in range(n):
   for j in range(n):
         sum = sum + 1
for p in range(0,n*n):
   for q in range(p):
      sum = sum - 1
for s in range(20):
   for t in range(5):
      sum = sum + 1
\end{lstlisting}
\begin{parts}
  \part[3] Using Big-O notation, What is the worst case performance
   for the set of loops in lines 1--3?
   \vspace{0.5in}
  \part[3] Using Big-O notation, What is the worst case performance
   for the set of loops in lines 4--6?
   \vspace{0.5in}
  \part[3] Using Big-O notation, What is the worst case performance
   for the set of loops in lines 7--9?
   \vspace{0.5in}
  \part[3] Using Big-O notation, What is the overall worst case performance?
   \vspace{0.5in}
\end{parts}

\newpage
\question[10]  In this question you must implement a Queue.  However you may not use a Python list, or the \texttt{Node} class.  You must use two \textbf{Stacks} to implement the Queue functionality.  Write the methods \verb~__init__~ \texttt{enqueue}, \texttt{dequeue}, and \texttt{isEmpty}.

\end{questions}


\end{document}

